%----------------------------------------------------------------------------------------
%	PACKAGES AND OTHER DOCUMENT CONFIGURATIONS
%----------------------------------------------------------------------------------------

\documentclass[final]{beamer}

% Добавляем поддержку русского языка
\usepackage[utf8]{inputenc}
\usepackage[T2A]{fontenc}
\usepackage[russian]{babel}

\usepackage[scale=1.24]{beamerposter} % Use the beamerposter package for laying out the poster

\usetheme{confposter} % Use the confposter theme supplied with this template

\setbeamercolor{block title}{fg=ngreen,bg=white} % Colors of the block titles
\setbeamercolor{block body}{fg=black,bg=white} % Colors of the body of blocks
\setbeamercolor{block alerted title}{fg=white,bg=dblue!70} % Colors of the highlighted block titles
\setbeamercolor{block alerted body}{fg=black,bg=dblue!10} % Colors of the body of highlighted blocks
% Many more colors are available for use in beamerthemeconfposter.sty

%-----------------------------------------------------------
% Define the column widths and overall poster size
% To set effective sepwid, onecolwid and twocolwid values, first choose how many columns you want and how much separation you want between columns
% In this template, the separation width chosen is 0.024 of the paper width and a 4-column layout
% onecolwid should therefore be (1-(# of columns+1)*sepwid)/# of columns e.g. (1-(4+1)*0.024)/4 = 0.22
% Set twocolwid to be (2*onecolwid)+sepwid = 0.464
% Set threecolwid to be (3*onecolwid)+2*sepwid = 0.708

\newlength{\sepwid}
\newlength{\onecolwid}
\newlength{\twocolwid}
\newlength{\threecolwid}
\setlength{\paperwidth}{48in} % A0 width: 46.8in
\setlength{\paperheight}{36in} % A0 height: 33.1in
\setlength{\sepwid}{0.024\paperwidth} % Separation width (white space) between columns
\setlength{\onecolwid}{0.22\paperwidth} % Width of one column
\setlength{\twocolwid}{0.464\paperwidth} % Width of two columns
\setlength{\threecolwid}{0.708\paperwidth} % Width of three columns
\setlength{\topmargin}{-0.5in} % Reduce the top margin size
%-----------------------------------------------------------

\usepackage{graphicx}  % Required for including images

\usepackage{booktabs} % Top and bottom rules for tables

%----------------------------------------------------------------------------------------
%	TITLE SECTION 
%----------------------------------------------------------------------------------------

\title{ГОСТДОК: Автоматизация форматирования документов по стандартам ГОСТ} % Poster title

\author{Студенческая веб-студия PolyWeb} % Author(s)

\institute{ФГАОУ ВО «Московский политехнический университет»} % Institution(s)

%----------------------------------------------------------------------------------------

\begin{document}

\addtobeamertemplate{block end}{}{\vspace*{2ex}} % White space under blocks
\addtobeamertemplate{block alerted end}{}{\vspace*{2ex}} % White space under highlighted (alert) blocks

\setlength{\belowcaptionskip}{2ex} % White space under figures
\setlength\belowdisplayshortskip{2ex} % White space under equations

\begin{frame}[t] % The whole poster is enclosed in one beamer frame

\begin{columns}[t] % The whole poster consists of three major columns, the second of which is split into two columns twice - the [t] option aligns each column's content to the top

\begin{column}{\sepwid}\end{column} % Empty spacer column

\begin{column}{\onecolwid} % The first column

%----------------------------------------------------------------------------------------
%	INTRODUCTION
%----------------------------------------------------------------------------------------

\begin{block}{Введение}

В современном мире эффективность и стандартизация документации являются важными аспектами успешной деятельности организаций. Многие компании и образовательные учреждения требуют строгого соблюдения стандартов оформления документов, таких как ГОСТ.

Ручное форматирование документов занимает много времени и часто приводит к ошибкам, что снижает производительность и может негативно сказываться на качестве.

\end{block}

%------------------------------------------------

\begin{block}{Актуальность проекта}

Многие пользователи (студенты, научные сотрудники, специалисты по документообороту) сталкиваются с проблемой оформления документов в соответствии с ГОСТ и другими стандартами.

Исследование стейкхолдеров выявило частые проблемы:
\begin{itemize}
\item Сложности с форматированием заголовков, таблиц, изображений
\item Ошибки компиляции и синтаксические ошибки в LaTeX
\item Необходимость строгого соблюдения стандартов
\end{itemize}

\end{block}

%----------------------------------------------------------------------------------------
%	OBJECTIVES
%----------------------------------------------------------------------------------------

\begin{alertblock}{Цели проекта}

Проект ГОСТДОК направлен на решение следующих задач:
\begin{itemize}
\item Создать сервис для автоматического форматирования документов по ГОСТ
\item Обеспечить доступ к высококачественной верстке для пользователей с разным уровнем подготовки
\item Решить проблемы, связанные с форматированием, ошибками компиляции и сложностью использования LaTeX
\end{itemize}

\end{alertblock}

%----------------------------------------------------------------------------------------

\end{column} % End of the first column

\begin{column}{\sepwid}\end{column} % Empty spacer column

\begin{column}{\twocolwid} % Begin a column which is two columns wide (column 2)

\begin{columns}[t,totalwidth=\twocolwid] % Split up the two columns wide column

\begin{column}{\onecolwid}\vspace{-.6in} % The first column within column 2 (column 2.1)

%----------------------------------------------------------------------------------------
%	MATERIALS
%----------------------------------------------------------------------------------------

\begin{block}{Суть проекта}

ГОСТДОК — веб-сервис для автоматического форматирования документов по стандартам ГОСТ и другим требуемым шаблонам с использованием LaTeX или через удобный веб-интерфейс.

Основные задачи проекта:
\begin{itemize}
\item Разработка веб-приложения с функционалом автоматического форматирования
\item Интеграция с LaTeX для обеспечения высококачественной верстки документов
\item Создание пользовательского интерфейса для удобства работы без опыта работы с LaTeX
\item Поддержка различных шаблонов и стандартов, включая ГОСТ
\end{itemize}

\end{block}

%----------------------------------------------------------------------------------------

\end{column} % End of column 2.1

\begin{column}{\onecolwid}\vspace{-.6in} % The second column within column 2 (column 2.2)

%----------------------------------------------------------------------------------------
%	METHODS
%----------------------------------------------------------------------------------------

\begin{block}{Этапы реализации}

Проект реализуется в несколько этапов:

\begin{enumerate}
\item Анализ требований и разработка концепции
\item Проведение исследования со стейкхолдерами и аналитики по проекту
\item Создание архитектуры веб-приложения
\item Разработка пользовательского интерфейса и интеграция с LaTeX
\item Тестирование и исправление ошибок
\item Запуск сервиса и получение обратной связи от пользователей
\end{enumerate}

\end{block}

%----------------------------------------------------------------------------------------

\end{column} % End of column 2.2

\end{columns} % End of the split of column 2 - any content after this will now take up 2 columns width

%----------------------------------------------------------------------------------------
%	IMPORTANT RESULT
%----------------------------------------------------------------------------------------

\begin{alertblock}{Промежуточный продуктовый результат}

На данном этапе реализации проекта достигнуты следующие ключевые результаты:
\begin{itemize}
\item Разработан базовый веб-интерфейс с регистрацией и управлением документами
\item Интегрирована возможность загрузки и компиляции текста в PDF с использованием LaTeX
\item Создан каталог шаблонов, поддерживающий стандарты ГОСТ
\item Добавлена поддержка вставки изображений, таблиц и работы с библиографией
\end{itemize}

\end{alertblock} 

%----------------------------------------------------------------------------------------

\begin{columns}[t,totalwidth=\twocolwid] % Split up the two columns wide column again

\begin{column}{\onecolwid} % The first column within column 2 (column 2.1)

%----------------------------------------------------------------------------------------
%	TECHNOLOGIES
%----------------------------------------------------------------------------------------

\begin{block}{Технологии}

Для реализации проекта используются следующие технологии:

\begin{itemize}
\item \textbf{Серверная часть:} Python, FastAPI
\item \textbf{Клиентская часть:} JavaScript, React
\item \textbf{Система верстки:} LaTeX
\end{itemize}

Архитектура системы спроектирована с учетом масштабируемости и удобства интеграции новых функций.

\end{block}

%----------------------------------------------------------------------------------------

\end{column} % End of column 2.1

\begin{column}{\onecolwid} % The second column within column 2 (column 2.2)

%----------------------------------------------------------------------------------------
%	RESULTS
%----------------------------------------------------------------------------------------

\begin{block}{Результаты}

\begin{table}
\vspace{2ex}
\begin{tabular}{l l}
\toprule
\textbf{Компонент} & \textbf{Статус}\\
\midrule
Концепция & Завершено \\
Документация & Завершено \\
Архитектура & Завершено \\
Серверная часть & В процессе \\
Пользовательский интерфейс & В процессе \\
Интеграция & В процессе \\
\bottomrule
\end{tabular}
\caption{Статус компонентов проекта}
\end{table}

\end{block}

%----------------------------------------------------------------------------------------

\end{column} % End of column 2.2

\end{columns} % End of the split of column 2

\end{column} % End of the second column

\begin{column}{\sepwid}\end{column} % Empty spacer column

\begin{column}{\onecolwid} % The third column

%----------------------------------------------------------------------------------------
%	FUNCTIONAL CAPABILITIES
%----------------------------------------------------------------------------------------

\begin{block}{Функциональные возможности}

Реализованные функции сервиса:
\begin{itemize}
\item Автоматическое форматирование документов по ГОСТ
\item Поддержка вставки изображений, таблиц и работы с библиографией
\item Функция автоматической проверки синтаксиса и предоставления подсказок
\item Каталог шаблонов для различных типов документов
\item Экспорт документов в формате PDF
\end{itemize}

\end{block}

%----------------------------------------------------------------------------------------
%	CONCLUSION
%----------------------------------------------------------------------------------------

\begin{block}{Заключение}

Проект ГОСТДОК успешно решает проблему автоматического форматирования документов в соответствии со стандартами ГОСТ. Достигнуты значительные успехи в анализе потребностей пользователей и разработке концепции сервиса.

Планы дальнейшей работы:
\begin{itemize}
\item Расширение функциональности приложения
\item Проведение тестирования для выявления и устранения ошибок
\item Оптимизация производительности
\item Подготовка итоговой документации и развертывание приложения
\end{itemize}

\end{block}

%----------------------------------------------------------------------------------------
%	REFERENCES
%----------------------------------------------------------------------------------------

\begin{block}{Список использованных источников}

\begin{enumerate}
\item Лампорт Л. LaTeX: система подготовки документов. Пользовательская документация. – М.: Вильямс, 2019. – 600 с.
\item Статьи и публикации на Overleaf.com. Современные практики использования LaTeX в веб-среде [Электронный ресурс]. – Режим доступа: https://www.overleaf.com/
\end{enumerate}

\end{block}

%----------------------------------------------------------------------------------------
%	ACKNOWLEDGEMENTS
%----------------------------------------------------------------------------------------

\setbeamercolor{block title}{fg=red,bg=white} % Change the block title color

\begin{block}{Благодарности}

\small{\rmfamily{Выражаем благодарность Московскому политехническому университету за поддержку проекта и предоставление ресурсов для его реализации.}} \\

\end{block}

%----------------------------------------------------------------------------------------
%	CONTACT INFORMATION
%----------------------------------------------------------------------------------------

\setbeamercolor{block alerted title}{fg=black,bg=norange} % Change the alert block title colors
\setbeamercolor{block alerted body}{fg=black,bg=white} % Change the alert block body colors

\begin{alertblock}{Контактная информация}

\begin{itemize}
\item Партнер проекта: ФГАОУ ВО «Московский политехнический университет»
\item Контактное лицо: Даньшина Марина Владимировна
\item Email: contact@gostdoc.ru
\item Веб: http://gostdoc.ru
\end{itemize}

\end{alertblock}

\begin{center}
\begin{tabular}{c}
\end{tabular}
\end{center}

%----------------------------------------------------------------------------------------

\end{column} % End of the third column

\end{columns} % End of all the columns in the poster

\end{frame} % End of the enclosing frame

\end{document}