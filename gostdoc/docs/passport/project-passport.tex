\documentclass{article}
\usepackage[utf8]{inputenc}  
\usepackage[T2A]{fontenc}    
\usepackage[russian]{babel}  
\usepackage[table,xcdraw]{xcolor}  
\usepackage{multirow}  
\usepackage{graphicx}  
\usepackage{adjustbox}  

\begin{document}

\begin{table}[]
\centering
\hspace*{-3.7cm} 
\resizebox{1.6\textwidth}{!}{
\begin{tabular}{|l|l|l|l|}
\hline
\multicolumn{2}{|l|}{\cellcolor[HTML]{F3F3F3}\textbf{Название проекта}} & \multicolumn{2}{l|}{PolyWeb} \\ \hline
\multicolumn{2}{|l|}{\cellcolor[HTML]{F3F3F3}\textbf{Название подпроекта (при наличии)}} & \multicolumn{2}{l|}{ГостДок} \\ \hline
\multicolumn{2}{|l|}{\cellcolor[HTML]{F3F3F3}\textbf{Заказчик}} & \multicolumn{2}{l|}{ФГАОУ ВО «Московский политехнический университет»} \\ \hline
\multicolumn{2}{|l|}{\cellcolor[HTML]{F3F3F3}\textbf{Контактное лицо со стороны заказчика}} & \multicolumn{2}{l|}{Даньшина Марина Владимировна} \\ \hline
\multicolumn{2}{|l|}{\cellcolor[HTML]{F3F3F3}\textbf{Контактный телефон}} & \multicolumn{2}{l|}{} \\ \hline
\multicolumn{2}{|l|}{\cellcolor[HTML]{F3F3F3}\textbf{Тип проекта}} & \multicolumn{2}{l|}{} \\ \hline
\multicolumn{2}{|l|}{\cellcolor[HTML]{F3F3F3}\textbf{Уровень проекта (TRL/УГТ) на начало уч. года}} & \multicolumn{2}{l|}{1} \\ \hline
\multicolumn{2}{|l|}{\cellcolor[HTML]{F3F3F3}\textbf{Проблематика проекта}} & \multicolumn{2}{p{10cm}|}{\raggedright 1. Необходимость ручного копирования и синхронизации одинаковых текстовых блоков в нескольких документах.\\ 2. Трудность соблюдения строгих стандартов ГОСТ.\\ 3. Сложность использования LaTeX, который требует специфических навыков и знаний.} \\ \hline
\multicolumn{2}{|l|}{\cellcolor[HTML]{F3F3F3}\textbf{Цель проекта}} & \multicolumn{2}{p{10cm}|}{Создать веб-сервис для автоматического форматирования множества документов по шаблонам или стандартам ГОСТ, используя модульный подход.} \\ \hline
\multicolumn{2}{|l|}{\cellcolor[HTML]{F3F3F3}\textbf{Продуктовый результат}} & \multicolumn{2}{l|}{Веб-приложение} \\ \hline
\multicolumn{2}{|l|}{\cellcolor[HTML]{F3F3F3}\textbf{Промежуточный продуктовый результат (на конец первого семестра для годовых проектов)}} & \multicolumn{2}{l|}{Прототип веб-приложения} \\ \hline
\multicolumn{2}{|l|}{\cellcolor[HTML]{F3F3F3}\textbf{Критерии достижения результата (основные характеристики продукта)}} & \multicolumn{2}{p{10cm}|}{\raggedright 1. Автоматическое форматирование документов.\\ 2. Компонентный подход.\\ 3. Синхронизация текстовых блоков. \\ 4. Интеграция с LaTeX \\ 5. Гибкость шаблонов} \\ \hline
\multicolumn{2}{|l|}{\cellcolor[HTML]{F3F3F3}\textbf{Ключевые задачи проекта (уникальные решения, необходимые для достижения результата)}} & \multicolumn{2}{p{10cm}|}{\raggedright 1. Реализовать компонентный подход.\\ 2. Интегрировать систему с LaTeX.} \\ \hline
\multicolumn{2}{|l|}{\cellcolor[HTML]{F3F3F3}\textbf{Дата старта проекта}} & \multicolumn{2}{l|}{16.09.2024} \\ \hline
\multicolumn{2}{|l|}{\cellcolor[HTML]{F3F3F3}\textbf{Длительность проекта}} & \multicolumn{2}{l|}{2 семестра} \\ \hline
\multicolumn{2}{|l|}{\cellcolor[HTML]{F3F3F3}\textbf{Количество студентов}} & \multicolumn{2}{l|}{22} \\ \hline
\multicolumn{2}{|l|}{\cellcolor[HTML]{F3F3F3}\textbf{Проектное направление}} & \multicolumn{2}{l|}{IT} \\ \hline
\multicolumn{2}{|l|}{\cellcolor[HTML]{F3F3F3}\textbf{Руководитель проектного направления}} & \multicolumn{2}{l|}{Стрижеус Валерий Александрович} \\ \hline
\multicolumn{2}{|l|}{\cellcolor[HTML]{F3F3F3}\textbf{Руководитель проекта}} & \multicolumn{2}{l|}{Даньшина Марина Владимировна} \\ \hline
\multicolumn{2}{|l|}{\cellcolor[HTML]{F3F3F3}\textbf{Кураторы проекта (подпроекта)}} & \multicolumn{2}{l|}{Даньшина Марина Владимировна} \\ \hline
\multicolumn{2}{|l|}{\cellcolor[HTML]{F3F3F3}\textbf{Бюджет проекта, руб.}} & \multicolumn{2}{l|}{0} \\ \hline
\multicolumn{2}{|l|}{\cellcolor[HTML]{F3F3F3}\textbf{– в т. ч. финансирование от ЦПД, руб.}} & \multicolumn{2}{l|}{0} \\ \hline
\multicolumn{2}{|l|}{\cellcolor[HTML]{F3F3F3}\textbf{– в т. ч. финансирование от заказчика, руб.}} & \multicolumn{2}{l|}{0} \\ \hline
\multicolumn{2}{|l|}{\cellcolor[HTML]{F3F3F3}\textbf{– доля участия заказчика, процент.}} & \multicolumn{2}{l|}{} \\ \hline
\end{tabular}%
}
\end{table}

\end{document}
